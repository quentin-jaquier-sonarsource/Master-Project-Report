\section{Conclusion}
\label{sec:conclusion}

We have described the use of a universal intermediate representation used at SonarSource to perform static analysis.
It has enabled the company to reduce the complexity of the ecosystem, the maintenance, and the overall effort to implement more than forty checks on a new language, without any strong push-backs.
With this representation, SonarSource managed to provide 5 new languages in less than a year, an effort that would not have been possible without it.

In addition to these forty checks, we pushed \slang{} further and managed to run a null pointer dereference checker on top of this incomplete representation. 
After some effort to adapt the language, the checker turns out to be as efficient as the implementation on the original intermediate representation.
During the implementation, the control flow graph showed some weak spots due to the presence of native nodes, but we managed to introduce the concept of unreliable basic block in order to report a good amount of issues with no obvious false positive.
The algorithm contains no complex elements and is still able to find multiple thousands of issues on existing open-source projects, on both Scala and Java.

Finally, we have looked at popular tools also performing null pointer dereference checks, we have identified multiple possible improvements, and pointed language features that can become limitations of the current approach.

\slang{} seems to have started in a very good way and has already demonstrated his power.
However, we have to keep in mind that the language is less than one year's old, if it suits well the situations SonarSource faces today, it still have many challenges to face.
